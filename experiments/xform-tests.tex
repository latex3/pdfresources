% !Mode:: "TeX:DE:UTF-8:Main"
\documentclass{article}
\usepackage{l3pdf,ifxetex,tikz}
\ExplSyntaxOn
\pdf_uncompress:
\ExplSyntaxOff
%\usepackage[generic]
%{attachfile2}
\begin{document}
%   \pbs_pdfxform:nnnnn
%     #1: add pgf/tikz resources (transparency, shading)? (1|0) %dvipdfmx/xetex
%     #2: used as PDF annotation appearance? (1|0)              %dvips/pdftex
%     #3: additional resources                                  %all BUT dvips
%     #4: additional dictionary entries
%     #5: savebox number

% xform: immediate or not?
% argument: a box.
% with dvipdfmx, xform should be in a 0-box.
% xform attr
% pdflatex: attr keyword
% xform resources
% pdflatex: resources keyword
% pdfbase current page resource + special stuff:
% \tl_set:Nx\l_tmpa_tl{\the\pdfpageresources~#3}\tl_trim_spaces:N\l_tmpa_tl

Some Text \tikz\fill[opacity=0.5,red](0,0)rectangle(1,1);

\newbox\testxform
\savebox\testxform{Some Text \tikz\fill[opacity=0.5,red](0,0)rectangle(1,1);}

\ExplSyntaxOn
\sys_if_engine_pdftex:T
{
 \wd\testxform=0.5\wd\testxform
 %Box before:\usebox\testxform \par %\showthe\pdfpageresources
 %saving doesn't occupy space, but as it is a whatsits ...
 % transparency needs local resources ...
  abc\immediate\pdfxform resources {\the\pdfpageresources}
     \testxform abc \par

  xxx\pdfrefxform \pdflastxform yyy \par

 Box after:\usebox\testxform \par

 %we need to resave the box to get the width ...
 \savebox\testxform{Some~Text \tikz\fill[opacity=0.5,red](0,0)rectangle(1,1);}
  \fbox{
 xx
  \__pdf_backend_annotation:nnnn
    {2\wd\testxform}
    {3\ht\testxform}
    {\dp\testxform}
    {
     /Subtype /Stamp
     /AP << /N~\the\pdflastxform\space0~R >>
    }
   }
}
\par


\ExplSyntaxOff

%\ifxetex %occupies space!
%xx\special{pdf:bxobj @MyStampA
%width 280pt height 0pt depth 40pt}
%%\tikz\draw[opacity=0.5](0,0)--(1,1);
%\tikz\fill[opacity=0.5,red](0,0)rectangle(1,1);
%
%My Own Stamp
% \special{pdf:put @resources~<</ExtGState @pgfextgs>>}
%\special{pdf:exobj}yy
%\fi

%\newpage

\ifxetex
\savebox\testxform{Some Text \tikz\fill[opacity=0.5,red](0,0)rectangle(1,1);}
% when saving the xform object, is should be in a zero box.
% all dimensions must be there, or the content is typeset ...
% and don't forget the \the!!
% transparency resources etc must be add to the box
abc%
\bigskip
Use xform: \special{pdf:uxobj @MyStamp}
%Use as appearance:
\fbox{
 xx
  \special
  {pdf:ann
   width  \the\dimexpr2\wd\testxform\relax\space
   height \the\dimexpr3\ht\testxform\relax\space
   depth  \the\dp\testxform
   <<
   /Type /Annot
   /Subtype /Stamp
   /AP << /N @MyStamp >>
   >>
   }
  \phantom{\usebox\testxform}xx}
abc 

 \begin{picture}(0,0)
  \put(0,0)
   {
    \special
     {pdf:bxobj @MyStamp
       width  \the\wd\testxform\space
       height \the\ht\testxform\space
       depth  \the\dp\testxform
      }%
    \usebox\testxform
   \special{pdf:put @resources <</ExtGState @pgfextgs>>} %
   \special{pdf:exobj}}
 \end{picture}yyyy

Use xform after: \special{pdf:uxobj @MyStamp}

\fbox{
 xx
  \special
  {pdf:ann
   width  \the\dimexpr2\wd\testxform\relax\space
   height \the\dimexpr3\ht\testxform\relax\space
   depth  \the\dp\testxform
   <<
   /Type /Annot
   /Subtype /Stamp
   /AP << /N @MyStamp >>
   >>
   }
  \phantom{\usebox\testxform}xx}
abc 

%\fi
%\end{document}


\fi
\end{document} 