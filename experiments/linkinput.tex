% !Mode:: "TeX:DE:UTF-8:Main"
\RequirePackage{pdfmanagement-testphase}
\DeclareDocumentMetadata{uncompress}
\RequirePackage{expl3}
\ExplSyntaxOn
\pdf_uncompress:
\ExplSyntaxOff
\documentclass{article}
\usepackage[T1]{fontenc}
\usepackage{hyperref}
\newcommand\cs[1]{\texttt{\textbackslash #1}}
\providecommand\pkg[1]{\texttt{#1}}
\makeatletter
{\catcode`\%=12 \gdef\hyper@space{%20}}
 \let\hyperspace\hyper@space
% \let\ \hyper@space
\makeatother
\begin{document}


\section{Some background about URL's and file references with hyperref}

hyperref has three commands related to URL and file references: \cs{url},
\cs{nolinkurl} and \cs{href}. The first two take one argument,
while the last has two, the url and some free text.


\cs{url} and \cs{href} create link annotations. \cs{url} creates always an URI
type, \cs{href} creates URI, GoToR and Launch depending on the structure of the argument.

\cs{href} has to create a (in the PDF) valid url or file name from its first argument.
\cs{url} has to create a (in the PDF) valid url from its only argument and has also to print
this argument as url. \cs{nolinkurl} only prints the url.

For the printing \cs{url} and \cs{nolinkurl} rely on the url package and its \cs{Url} command.

(Expandable) commands are expanded and special chars can also be input by commands but
beside this no conversion is done: for all input hyperref basically assumes that
the input is already a valid percent encoded url or a valid file name. hyperref also
doesn't extend or add protocols.

As nowadays everyone is used to copy and paste links with all sorts of unicode into a browser and
they work the \pkg{hyperref} input is clearly rather restricted.


With \cs{href} it is possible to extend the input methods and to allow unicode input which is then
percent encoded by the code.

But extending the \emph{print} options for \cs{url} and \cs{nolinkurl}
is hard to impossible in pdf\LaTeX{} due to the way the url package works.
Some chars can be added with the help of \cs{UrlSpecial} (at the cost of warnings)
but it doesn't work for every input and documenting and explaining all the edge cases is no joy.

\section{Links to files}

When a file is linked with \cs{href} than normally it is added as URI link. The exceptions are PDF's:
for them PDF has the special type GoToR which allows also to link to a destination or a special page.

After a number of tests with various PDF viewer established that non-ascii files names don't
work at all with a simple file name specification GoToR links now use a full
filespec dictionary. This works better, but still no every PDF viewer support this correctly.
on various system.

The following can be used to test viewers. It assumes that a \texttt{test.pdf},
a \texttt{grüßpdf.pdf} and a \texttt{grüße.txt} are in the current folder.

\hrefpdf{test.pdf}{test-ascii}

\hrefpdf{grüßpdf.pdf}{test grüßpdf.pdf}

\hrefurl[urlencode]{grüße.txt}{test grüße.txt}


\section{Splits}

\cs{href} tries to be clever and to detect from the argument
if a url or a file link or a launch command should be created.

The rules are not trivial, and they make the code complicated.
This detection also makes it more difficult to handle special cases
like non-ascii input for the link types.

For this reason three new commands have been create:

\begin{itemize}
\item \cs{hrefurl} for standard urls (and non-pdf files)
\item \cs{hrefpdf} for references to pdf files
\item \cs{hrefrun} for launch links
\end{itemize}

The new commands don't use prefixes like \cs{href}.
Their argument should be the real content.

All commands have an option argument for keyval syntax.
It accepts the following keys. Not all keys make sense for all keys, but they don't
error, they are silently ignored.

\begin{tabular}{llp{6cm}}
key & applicable commands  & note\\\hline
urlencode & \cs{hrefurl}   & if set the code will convert the argument to percent encoding. This allows non-ascii input.\\
protocol &  \cs{hrefurl}, \cs{url} & This sets a prefix/protocol that is added to the url, see below. \\
format   & \cs{url}   & a command used to format the printed text. It replaces the standard \cs{Url}. This can improve non-ascii
typesetting at the cost of losing the special line breaking.\\
destination &  \cs{href}, \cs{hrefpdf} & A destination name in the PDF\\
page & \cs{href}, \cs{hrefpdf}        & destination page, default: 1\\
pdfremotestartview &\cs{href}, \cs{hrefpdf} & start view, default: Fit\\
ismap & \cs{href}, \cs{hrefurl}       & see PDF reference\\
afrelationship &  \cs{href}, \cs{hrefpdf} & Changes the /AFRelationship key of the filespec dictionary. The value should be a PDF name without the starting slash.\\
run-parameter & \cs{hreflaunch} & run parameter (see the PDF reference)\\
nextactionraw & various & puts a /Next entry in the action dictionary (see the PDF reference)\\
\end{tabular}


The first four keys can be set also in \cs{hypersetup} for all following commands in
the current group through the keys
\texttt{href/urlencode}, \texttt{href/protocol}, \texttt{href/destination}, \texttt{href/format}.

It is possible to define own url commands with specific options e.g. with

\begin{verbatim}
\NewDocumentCommand\myurl{O{}}{\url[protocol=https://,format=\textsc,#1]}
\end{verbatim}




\begin{verbatim}
href creates different link types.

If there is a colon :
  -> if prefix = file: GoToR
  -> if prefix = run:  Launch
  -> if prefix empty:  GoToR
  -> rest: URI

If there is no colon:
  if there is a dot:
    -> extension: pdf  GoToR
    -> else URI
  if there is no dot: GoToR

url creates always URI

See
  -> \@hyper@readexternallink
  -> \@hyper@linkfile
\end{verbatim}

\url{run://runuri}{xxx}

\href{:emptyprefixGoToR}{xxx}

\href{file:fileprefixGoToR}{xxx}

\href{run:runprefixLaunch}{xxx}

\href{blub:blubprefixURI}{xxx}

\href{nocolonnodotGoToR}{xxx}

\href{colonURI.dot}{xxx}

\href{colonGoToR.pdf}{xxx}


\section{String}

\def\blub{file://www.teststrings.de}
\href{\blub}{aaa}
\ExplSyntaxOn %no underscores in command names ...
\str_set:Nx\ltmpastr{file\c_colon_str//www.teststrings.de\c_percent_str \c_hash_str 123}
\href{\ltmpastr}{blub}
\hrefurl{\ltmpastr}{blub}
\hrefpdf{\ltmpastr}{blub}
\ExplSyntaxOff


\section{parentheses}
These are correctly escaped
% URI(https://www.xyz.de\(blub\))
\href{https://www.xyz.de(blub)}{xxx}
\url{https://www.xyz.de(blub)}

\href{https://www.xyz.de(blub}{xxx}
\url{https://www.xyz.de(blub}

%\end{document}


\section{linebreaks}

% gives https://www.xyz\040zzz.de
\href{https://www.xyz
 zzzlinebreak.de}{xxx}
\url{https://www.xyz
zzzlinebreak.de}

% ok
\href{https://www.xyz%
zzzlinebreak.de}{xxx}
\url{https://www.xyz%
zzzlinebreak.de}


\section{tilde}
%\makeatletter \def\hyper@tilde{X}\makeatother
\href{https://www.xyz.de~blub}{xxx}
\url{https://www.xyz.de~blub}

\href{https://www.xyz.de\~blub}{xxx}
\url{https://www.xyz.de\~blub}

\makeatletter
\href{https://www.xyz.de\textasciitilde blub}{xxx}
\url{https://www.xyz.de\textasciitilde blub}
\makeatother



\section{space}
Print output is wrong, uri in pdf too (but this works with pdfmanagement).
\href{https://www.xyz.de blub}{xxx}
\url{https://www.xyz.de blub}

\href[urlencode]{https://www.xyz.de blub}{xxx}
\url[urlencode]{https://www.xyz.de blub}

\hrefurl[urlencode]{https://www.spacehrefurl.de blub}{xxx}

\parbox{5cm}{\url{https://www.xyz.de\space blub}}


\section{percentchar}
The percent is not escaped, but in pdfmanagement it is!!
\href{https://www.testpercent.de%blub}{xxx}
\href{https://www.testpercentcmd.de\%blub}{xxx}
\url{https://www.testpercent.de%blub}
\url{https://www.testpercentcmd.de\%blub}
\hrefurl{https://www.testpercent.de%blub}{xxx}
\hrefurl{https://www.testpercentcmd.de\%blub}{xxx}


\makeatletter
\href{https://www.xyz.de\@percentchar blub}{xxx}
\url{https://www.xyz.de\@percentchar blub}
\parbox{3cm}{\url{https://www.xyz.de\@percentchar blub}}
\makeatother

\section{non ascii}
Is expanded to commands
\href[urlencode]{https://www.nonascii.deöxxx}{xxx}
\href{filewithnonasciiöäüxxx}{xxx}
%\url{https://www.xyz.deöxxx}

%loops
%\href{https://www.xyz.de§xxx}{xxx}
%\url{https://www.xyz.de§xxx}

Works more or less but is nonsense and errors with
pdfmanagement.
\fontencoding{T1}\selectfont
\href{filewithnonasciiöäüxxx}{xxx}
%\href{https://www.xyz.deöxxx}{xxx}
%\url{https://www.xyz.deöxxx}

% expanded to \\T1\\ss\040
%\href{https://www.xyz.deßxxx}{xxx}
%\url{https://www.xyz.deßxxx}


%\section{simple command}
\def\blub{abc}
\href{https://www.xyz.de\blub xxx}{xxx}
\url{https://www.xyz.de\blub xxx}


%\section{backslash}
In pdf it is doubled! \verb+\textbackslash+ loops.
\href{https://www.xyz.de\\blub}{xxx}
\url{https://www.xyz.de\\blub}
%\href{https://www.xyz.de\textbackslash blub}{xxx}
%\url{https://www.xyz.de\textbackslash blub}

\makeatletter
\href{https://www.xyz.de\@backslashchar blub}{xxx}
\url{https://www.xyz.de\@backslashchar blub}

\makeatother


\section{Tilde}

\href{https://www.xyztilde.de~blub}{xxx}
\href{https://www.xyzcmdtilde.de\~blub}{xxx}
\url{https://www.xyztilde.de~blub}
\url{https://www.xyzcmdtilde.de\~blub}
\href{https://www.xyztexttilde.de\textasciitilde blub}{xxx}
\url{https://www.xyztexttilde.de\textasciitilde blub}

\makeatletter
\href{https://www.xyzhypertilde.de\hyper@tilde blub}{xxx}
\url{https://www.xyzhypertilde.de\hyper@tilde blub}

\makeatother



\section{Math shift}
only direct input works
\href{https://www.xyzmathshift.de$blub}{xxx}
\url{https://www.xyzmathshift.de$lub}



\section{Ambersand}

\href{https://www.xyz.de&blub}{xxx}
\href{https://www.xyz.de\&blub}{xxx}
\url{https://www.xyz.de&blub}
\url{https://www.xyz.de\&blub}



\section{underscore}

\href{https://www.xyz.de_blub}{xxx}
\href{https://www.xyz.de\_blub}{xxx}
\url{https://www.xyz.de_blub}
\url{https://www.xyz.de\_blub}

\href{https://www.xyz.de\textunderscore blub}{xxx}
\url{https://www.xyz.de\textunderscore blub}





\section{hash}

\href{https://www.xyz.de#blub}{xxx}
\href{https://www.xyz.de\#blub}{xxx}
\href{filehash#blub}{xxx}
\url{https://www.xyz.de#blub}
\url{https://www.xyz.de\#blub}

\makeatletter
\href{https://www.xyz.de\hyper@hash blub}{xxx}
\url{https://www.xyz.de\hyper@hash blub}
\makeatother



\end{document} 