% !Mode:: "TeX:DE:UTF-8:Main"
\RequirePackage{pdfmanagement-testphase}
\DeclareDocumentMetadata{uncompress}
\RequirePackage{expl3}
\ExplSyntaxOn
\pdf_uncompress:
\ExplSyntaxOff
\documentclass{article}
\usepackage[T1]{fontenc}
\usepackage{hyperref}
\newcommand\cs[1]{\texttt{\textbackslash #1}}
\begin{document}

\section{Some background about URL's with hyperref}

hyperref has three commands related to URL: \cs{url},
\cs{nolinkurl} and \cs{href}. The first two take one argument,
while the last has two, the url and some free text.


\cs{url} and \cs{href} create link annotations. \cs{url} creates always an URI 
type, \cs{href} create URI, GoToR and Launch depending on the structure of the argument.

\cs{href} has to create a (in the PDF) valid url or file name from its first argument.
\cs{url} has to create a (in the PDF) valid url from its first argument and has to print 
the url. \cs{nolinkurl} only prints the url.

For the printing \cs{url} and \cs{nolinkurl} rely on the url package and its \cs{Url} command.

(Expandable) commands are expanded and special chars can also be input by commands but beside this
no conversion is done: for all input hyperref basically assumes that the input is a valid percent encoded url or
a valid file name.

%\begingroup
%\makeatletter
%\def\url@#1{\hyper@linkurl{§#1}{xxxx#1}}
%\url{www.blub.de}
%\endgroup
%\ExplSyntaxOn\makeatletter
%\begingroup
%%\let\protect\@unexpandable@protect
%\def\blub{abc}
%\__hyp_text_pdfstring:noN
%           { §x\blub xx}
%           { \l__hyp_text_enc_uri_print_tl }
%           \l__hyp_uri_tmpa_tl
% \tl_show:N \l__hyp_uri_tmpa_tl
%\ExplSyntaxOff
%\endgroup
%\end{document}

\section{string}

\ExplSyntaxOn
\str_set:Nx\l_tmpa_str{file://www.abcstring.de\c_percent_str \c_hash_str 123}
\expandafter\href\expandafter{\l_tmpa_str}{blub}
\ExplSyntaxOff
\section{splits}

\begin{verbatim}
href creates different link types.

If there is a colon :
  -> if prefix = file: GoToR
  -> if prefix = run:  Launch
  -> if prefix empty:  GoToR
  -> rest: URI

If there is no colon:
  if there is a dot:
    -> extension: pdf  GoToR
    -> else URI
  if there is no dot: GoToR

url creates always URI

See
  -> \@hyper@readexternallink
  -> \@hyper@linkfile
\end{verbatim}  

\url{run://runuri}{xxx}

\href{:emptyprefixGoToR}{xxx}

\href{file:fileprefixGoToR}{xxx}

\href{run:runprefixLaunch}{xxx}

\href{blub:blubprefixURI}{xxx}

\href{nocolonnodotGoToR}{xxx}

\href{colonURI.dot}{xxx}

\href{colonGoToR.pdf}{xxx}

%\end{document}


\section{parentheses}
These are correctly escaped
% URI(https://www.xyz.de\(blub\))
\href{https://www.xyz.de(blub)}{xxx}
\url{https://www.xyz.de(blub)}

\href{https://www.xyz.de(blub}{xxx}
\url{https://www.xyz.de(blub}

%\end{document}


\section{linebreaks}

% gives https://www.xyz\040zzz.de
\href{https://www.xyz
 zzzlinebreak.de}{xxx}
\url{https://www.xyz
zzzlinebreak.de}

% ok
\href{https://www.xyz%
zzzlinebreak.de}{xxx}
\url{https://www.xyz%
zzzlinebreak.de}

%\end{document}

\section{tilde}
%\makeatletter \def\hyper@tilde{X}\makeatother
\href{https://www.xyz.de~blub}{xxx}
\url{https://www.xyz.de~blub}

\href{https://www.xyz.de\~blub}{xxx}
\url{https://www.xyz.de\~blub}

\makeatletter
\href{https://www.xyz.de\textasciitilde blub}{xxx}
\url{https://www.xyz.de\textasciitilde blub}
\makeatother
%\end{document}


\section{space}
Print output is wrong, uri in pdf too (but this works with pdfmanagement).
\href{https://www.xyz.de blub}{xxx}
\url{https://www.xyz.de blub}

\href{https://www.xyz.de\space blub}{xxx}
\url{https://www.xyz.de\space blub}

\parbox{5cm}{\url{https://www.xyz.de\space blub}}

%\end{document}
\section{percentchar}
The percent is not escaped, but in pdfmanagement it is!!
\href{https://www.xyz.de%blub}{xxx}
\href{https://www.xyz.de\%blub}{xxx}
\url{https://www.xyz.de%blub}
\url{https://www.xyz.de\%blub}

\makeatletter
\href{https://www.xyz.de\@percentchar blub}{xxx}
\url{https://www.xyz.de\@percentchar blub}
\parbox{3cm}{\url{https://www.xyz.de\@percentchar blub}}


\makeatother
\section{non ascii}
Is expanded to commands
\href{https://www.xyz.deöxxx}{xxx}
\url{https://www.xyz.deöxxx}

%loops
%\href{https://www.xyz.de§xxx}{xxx}
%\url{https://www.xyz.de§xxx}

Works more or less but is nonsense and errors with
pdfmanagement. 
\fontencoding{T1}\selectfont
\href{https://www.xyz.deöxxx}{xxx}
\url{https://www.xyz.deöxxx}

% expanded to \\T1\\ss\040
\href{https://www.xyz.deßxxx}{xxx}
\url{https://www.xyz.deßxxx}


%\section{simple command}
\def\blub{abc}
\href{https://www.xyz.de\blub xxx}{xxx}
\url{https://www.xyz.de\blub xxx}


%\section{backslash}
In pdf it is doubled! \verb+\textbackslash+ loops.
\href{https://www.xyz.de\\blub}{xxx}
\url{https://www.xyz.de\\blub}
%\href{https://www.xyz.de\textbackslash blub}{xxx}
%\url{https://www.xyz.de\textbackslash blub}

\makeatletter
\href{https://www.xyz.de\@backslashchar blub}{xxx}
\url{https://www.xyz.de\@backslashchar blub}

\makeatother


\section{Tilde}

\href{https://www.xyztilde.de~blub}{xxx}
\href{https://www.xyzcmdtilde.de\~blub}{xxx}
\url{https://www.xyztilde.de~blub}
\url{https://www.xyzcmdtilde.de\~blub}
\href{https://www.xyztexttilde.de\textasciitilde blub}{xxx}
\url{https://www.xyztexttilde.de\textasciitilde blub}

\makeatletter
\href{https://www.xyzhypertilde.de\hyper@tilde blub}{xxx}
\url{https://www.xyzhypertilde.de\hyper@tilde blub}

\makeatother



\section{Math shift}
only direct input works
\href{https://www.xyzmathshift.de$blub}{xxx}
\url{https://www.xyzmathshift.de$lub}



\section{Ambersand}

\href{https://www.xyz.de&blub}{xxx}
\href{https://www.xyz.de\&blub}{xxx}
\url{https://www.xyz.de&blub}
\url{https://www.xyz.de\&blub}



\section{underscore}

\href{https://www.xyz.de_blub}{xxx}
\href{https://www.xyz.de\_blub}{xxx}
\url{https://www.xyz.de_blub}
\url{https://www.xyz.de\_blub}

\href{https://www.xyz.de\textunderscore blub}{xxx}
\url{https://www.xyz.de\textunderscore blub}





\section{hash}

\href{https://www.xyz.de#blub}{xxx}
\href{https://www.xyz.de\#blub}{xxx}
\url{https://www.xyz.de#blub}
\url{https://www.xyz.de\#blub}

\makeatletter
\href{https://www.xyz.de\hyper@hash blub}{xxx}
\url{https://www.xyz.de\hyper@hash blub}
\makeatother



\end{document} 